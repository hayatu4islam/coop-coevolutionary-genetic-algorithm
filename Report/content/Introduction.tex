\section{Introduction} \label{sec:introduction}

The third paper, \textit{A Cooperative Coevolutionary Approach to Function Optimisation}\cite{original-paper} was chosen for this assignment.
The paper presents a framework for a Cooperative Coevolutionary Genetic Algorithm (CCGA) and compares its performance with a standard Genetic Algorithm (GA) in function optimisation problem.

The functions were chosen for being \textit{``highly multi-modal''}, presenting a difficult challenge for a hill-climber.
These functions are referred to as the Rastrigin, Schwefel, Griewangk, and Ackley functions\cite{functions-1,functions-2,functions-3}.
Each of these functions has a global minimum of zero when the set of parameters fed into it are bounded by set values.
The optimisation task presented in the paper is to evolve the set of parameters that give an output of zero.
Each algorithm's performance is tested on each function individually.

In the standard GA, the individuals being evolved are bit-strings representing every parameter.
The fitness of each individual is calculated by by splitting the bit-string into the parameters and running it through the function.
In the CCGA, each parameter is represented by its own population of bit-strings.
Each population is evaluated in a round-robin fashion.
The fitness of each individual is calculated by combining it with the best individuals from each of the other populations and running this set through the function.
An initial fitness value is assigned before the first generation by evaluating each individual with a random individual from each population.
In both algorithms, the closer the function output is to zero, the fitter the individual.

A scaling window is used to translate this smaller-is-better fitness regime into a fitness proportionate selection scheme.
When calculating the selection probability of an individual its fitness is subtracted from the fitness of the worst individual from the past 5 generations.
This value is used when calculating the size of an individuals 'slice' on the roulette wheel.
This also allows small variations in the population to standout relative to the rest of the population.

Table \ref{tab:parameter-table}, reproduced from the original paper, contains the algorithm characteristics used in the experiments.
Pseudo code for the algorithms implemented can be found in \cite{original-paper}.


\begin{table} [h]
    \begin{tabular}{|l|l|}
        \hline
        \textbf{Characteristic}      & \textbf{Value}                       \\
        \hline
        Representation          & Binary (16 bits per function parameter)   \\
        Selection               & Fitness Proportionate                     \\
        Fitness Scaling         & Scaling Window Technique (width 5)        \\
        Elitist Strategy        & Single copy of best individual preserved  \\
        Genetic Operators       & Two-point crossover and bit-flip mutation \\
        Mutation Probability    & 1/chromlength                             \\
        Crossover Probability   & 0.6                                       \\
        Population Size         & 100                                       \\
        Simulation Length       & 100000 function evaluations               \\
        \hline

    \end{tabular}
    \caption{A table showing the characteristics of the algorithms used to perform the experiments. Reproduced from \cite{original-paper}}
    \label{tab:parameter-table}
\end{table}
