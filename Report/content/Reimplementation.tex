\section{Reimplementation} \label{sec:reimplementation}

The reimplementation was written in Python v3.8 for flexibility and speed of development.
The results generated by the Python script were saved to disk and plotted using MATLAB.
This meant plots could be fine tuned without having to rerun the experiments.

Individuals are defined using an \texttt{Individual} class in \texttt{individual.py} (Appendix \ref{lst:individual}). 
This class defines the genome as a BitArray object (defined in the third-party bistring library\cite{bistring-repo}), and that individual's fitness for convenience.
The fitness functions are defined in \texttt{functions.py} (Appendix \ref{lst:functions}). 
Each function is defined with identical interfaces and an accompanying Python dict containing details about the function such as the limits of the parameter values.
Also included are functions to convert a set of BitArrays to function parameters.
The original paper did not specify the conversion process used so the range of $[0, 2^{16} - 1]$ that the BitArray can represent is mapped to the lower and upper limits of the functions (e.g. $[-5.12, 5.12]$ for the Rastrigin function).

The algorithms are implemented in the classes \texttt{GAExperiment} and \texttt{CCGAExperiment} (see Appendices \ref{lst:ga_experiment}, \ref{lst:ccga_experiment}). 
These classes implement generic versions of their algorithm and must be provided with the fitness function, the corresponding Python dict, and the number of parameters under test upon instantiation.
The experiment can then be run using the \texttt{run\_experiment()} method.
The \texttt{CCGAExperiment} class shares the same overall structure as the \texttt{GAExperiment} class but has been upgraded with the infrastructure needed to store, evolve, and evaluate multiple populations.

The experiments are performed in the file \texttt{main\_data\_gather.py} (see Appendix \ref{lst:main_data_gather}).
This acts as the project's main file.
The user specifies which algorithms should be tested and the number of runs the results should be averaged over with command line arguments.
The results are saved to txt files for plotting.
The computing resources available to this project were limited due to working from home so results of each experiment were averaged over 15 runs rather than the 50 in the original paper.

The MATLAB script \texttt{combined\_plots.m} (see Appendix \ref{lst:combined_plots}) was used to produce the reimplemented figure.
